\documentclass[10pt, a4paper]{article}

% Parametri che modificano il file main.tex
% Le uniche parti da cambiare su main.tex sono:
% - vari \vspace tra sezioni
% - tabella azioni da intraprendere
% - sezione altro

\def\data{2023-11-15}
\def\oraInizio{15:00}
\def\oraFine{16:00}
\def\luogo{Piattaforma Google Meet}

\def\tipoVerb{Esterno} % Interno - Esterno

\def\nomeResp{Feltrin E.} % Cognome N.
\def\nomeVer{Campese M.} % Cognome N.
\def\nomeSegr{Orlandi G..} % Cognome N.

\def\nomeAzienda{Azzurro Digitale}
\def\firmaAzienda{azzurrodigitale.png}
\def\firmaResp{emanuele.png} % nome Responsabile

\def\listaPartInt{
Bresolin G.,
Campese M.,
Ciriolo I.,
Dugo A.,
Feltrin E.,
Michelon R.,
Orlandi G.
}

\def\listaPartEst{
\textit{Azzurro Digitale:},
Caliendo G.,
D'Avanzo C.}

}

\def\listaRevisioneAzioni {x}

\def\listaOrdineGiorno {
{Riunione con l'azienda AzzurroDigitale;},
{L'azienda ci ha esposto i requisiti che deve avere il progetto e ci hanno fatto una breve introduzione per discutere in modo generale , cosa si aspettano il software debba saper fare;},
{Possibilità di porre alcune domande generiche sul progetto ai referenti Caliendo G. e D'Avanzo C., con seguente approfondimento dei vari temi toccati.\\ .}
}

\def\listaDiscussioneInterna {
{Discussione 1;},
{Discussione 2;},
}

\newcommand{\domris}[2]{\textbf{#1}\\#2}

\def\listaDiscussioneEsterna {
\domris
{In generale , quali sono i requisiti funzionali fondamentali che deve avere il sistema?
}
{L'azienda ci ha risposto , che certamente la ricezione in input e la gestione dei documenti , deve essere efficiente;
},
\domris
{Quali sono le funzioni principali per il caricamento dei documenti?
}
{L'azienda ha detto che come minimo , il sistema dovrà essere capace di ricevere in input ed elaborare documenti Pdf e documenti Word;
},
\domris
{I documenti caricati dovranno rimanere accessibili?
}
{L'azienda ci ha risposto che i documenti caricati potranno essere accessibili , ma che non serviranno grandi cose , basta un catalogo ;
},
\domris
{Quali funzioni serviranno per eliminare i documenti? Il modello dovrà essere riaddestrato?
}
{L'azienda ci ha risposto che se un documento subisce modifiche , il sistema dovrà evolvere ed essere riaddestrato;
},
\domris
{L'interazione tra l'utente e il chatbot , quali saranno i protocolli? 
}
{L'azienda ci ha risposto , che per quanto riguarda il comando vocale , da parte dei lavoratori non sarà fondamentale , ma che come minimo se un lavoratore chiede cose errate , il sistema dovrà dirglielo e che possa memorizzare una storia delle chat tra il lavoratore ed esso}.
}
} 




\def\listaDecisioni {x}

\usepackage{style}
\usepackage{headerfooter}

\title{\titolo}
\author{SWEetCode}

\begin{document}

% PRIMA PAGINA
\include{firstpage}

% REGISTRO DELLE VERSIONI
%Registro in ordine dalla più recente alla meno recente!
{\renewcommand{\arraystretch}{1.5}
\section*{Registro delle versioni}

\begin{xltabular}{\textwidth}{c|c|c|c|X}
\label{tab:long}

\textbf{Versione} & \textbf{Data} & \quantities{\textbf{Responsabile di}\\\textbf{stesura}}& \textbf{Revisore} & \quantities{\textbf{Dettaglio e}\\\textbf{motivazioni}} \\
\endfirsthead

\textbf{Versione} & \textbf{Data} & \quantities{\textbf{Responsabile di}\\\textbf{stesura}}& \textbf{Revisore} & \quantities{\textbf{Dettaglio e}\\\textbf{motivazioni}} \\
\endhead

\multicolumn{5}{r}{{Continua nella pagina successiva}} \\
\endfoot

\endlastfoot

\hline
v2.0.0(1) & $2023-11-18$ & Michelon R. & Campese M. & Aggiornata introduzione.\\
\hline
v1.0.0(8) & $2023-11-18$ & Michelon R. & Campese M. & Cambiata introduzione.\\
\hline
v1.0.0(6) & $2023-11-17$ & Dugo A. & Riccardo & modifica per prova1.\\
\hline
v1.0.0(5) & $2023-11-17$ & Dugo A. & Campese M. & registro delle modifiche prova.\\
\hline
v1.0.1 & $2023-11-14$ & \quantities{Ciriolo I.} & Campese M. &  Studio dei primi casi d'uso.\\
\hline
v0.0.1 & $2023-11-06$ & \quantities{Ciriolo I.\\Orlandi G.} & Campese M. &  Impostazione del documento, Introduzione.\\
    
\end{xltabular}}
\newpage

% INDICE
\tableofcontents
\newpage

% INTRODUZIONE
\section{Introduzione}
\subsection{Scopo del documento}
Il Piano di Qualifica è un documento che si valuta di modificare incrementalmente, in particolare per 
definire le metriche di valutazione del prodotto, che saranno definite conformemente ai requisiti e 
alle aspettative del proponente, al fine di poter correttamente definire la qualità del prodotto, 
attraverso un processo di miglioramento continuo e che, per sua natura, tende a diventare 
incrementale nel corso del tempo e quando viene definita baseline. Per tutti questi motivi, la qualità 
viene definita da un insieme di processi che cerchino di definire metriche di misurazione di efficacia 
ed efficienza (misure quantitative che serviranno da valutazione nel corso di realizzazione del 
progetto didattico)
\paragraph{}
%\subsection{Scopo del prodotto}
\subsection{Glossario}
Al fine di evitare incomprensioni relative alla terminologia usata all’interno del documento, viene 
fornito un Glossario nel file apposito, tale da non avere terminologie ambigue nell’attività 
progettuale individuata e dandone una definizione precisa. Ogni termine avrà nel documento una 
lettera G come apice, per meglio evidenziare la loro appartenenza al documento indicato.

\subsection{Riferimenti}
   \subsubsection{Riferimenti normativi}
   \begin{itemize}
    \item \textit{Regolamento del progetto didattico}: \\
    \href{https://www.math.unipd.it/~tullio/IS-1/2023/Dispense/PD2.pdf}{https://www.math.unipd.it/~tullio/IS-1/2023/Dispense/PD2.pdf}\\
    \item \textit{Norme di progetto v0.4.3};
    \end{itemize}
    
    \subsubsection{Riferimenti informativi}
    
    \begin{itemize}
    \item \textit{Glossario v0.0.1} (da creare parallelamente) 
    \item \textit{Presentazione capitolato C1}:\\
    \href{https://www.math.unipd.it/~tullio/IS-1/2023/Progetto/C1.pdf}{https://www.math.unipd.it/~tullio/IS-1/2023/Progetto/C1.pdf}
    \item \textit{Verbali esterni ed interni};
    \item \textit{Analisi e descrizione delle funzionalità: Use Case e relativi diagrammi (UML)}:\\
    \href{https://www.math.unipd.it/~rcardin/swea/2022/Diagrammi%20Use%20Case.pdf}{https://www.math.unipd.it/~rcardin/swea/2022/Diagrammi\%20Use\%20Case.pdf}
    \item \textit{Analisi e descrizione delle funzionalità: Diagrammi delle Attività (UML)}:\\
    \href{https://www.math.unipd.it/~rcardin/swea/2022/Diagrammi%20di%20Attivit%C3%A0.pdf}{https://www.math.unipd.it/~rcardin/swea/2022/Diagrammi\%20di\%20Attivit\%C3\%A0.pdf}
    \item \textit{Progettazione e Programmazione: Diagrammi delle Classi}:\\
    \href{https://www.math.unipd.it/~rcardin/swea/2023/Diagrammi%20delle%20Classi.pdf}{https://www.math.unipd.it/~rcardin/swea/2023/Diagrammi\%20delle\%20Classi.pdf}
    \end{itemize}

% Qualità di Processo
\newpage
\section{Qualità di processo}
\subsection{Scopo ed obiettivi}
La qualità è determinata univocamente dai processi che lo compongono, misurata mettendo in atto 
delle metriche che permettano di valutare tali processi e accertarsi che raggiungano i corretti 
obiettivi di qualità previsti. In particolare, si fa riferimento al cosiddetto Ciclo PDCA (Plan - Do - Check 
- Act), atto a garantire un miglioramento continuo nell’utilizzo dei processi e delle risorse tramite 
pianificazione, successiva verifica con le metriche previste ed integrazione in fase di produzione. Di 
seguito, i processi individuati e i livelli di qualità previsti per ciascuno.
In particolare, per ciascuna metrica si opera una breve descrizione, dando un’idea comprensiva 
dell’attuazione e dei valori considerati accettabili in fase di controllo (check) qualità

\subsection{Processi primari}
    {\renewcommand{\arraystretch}{1.5}
    \begin{tabularx}{\textwidth}{p{0.18\textwidth}|p{0.6\textwidth}|X}
    \textbf{Obiettivo} & \textbf{Descrizione} & \textbf{Metriche}  \\
    \hline
    Fornitura &  & \\
    \hline
    Sviluppo &  &  \\
    \end{tabularx}}

\subsection{Processi di supporto}
    {\renewcommand{\arraystretch}{1.5}
    \begin{tabularx}{\textwidth}{p{0.18\textwidth}|p{0.6\textwidth}|X}
    \textbf{Obiettivo} & \textbf{Descrizione} & \textbf{Metriche}  \\
    \hline
    Verifica &  &  \\
    \hline
    Gestione della qualità &  &  \\
    \end{tabularx}}
    
\subsection{Processi organizzativi}
    {\renewcommand{\arraystretch}{1.5}
    \begin{tabularx}{\textwidth}{p{0.18\textwidth}|p{0.6\textwidth}|X}
    \textbf{Obiettivo} & \textbf{Descrizione} & \textbf{Metriche}  \\
    \hline
    Gestione organizzativa &  &  \\
    \end{tabularx}}
    
\subsection{Metriche utilizzate}
\subsubsection{Processi primari}
    {\renewcommand{\arraystretch}{1.5}
    \begin{tabularx}{\textwidth}{p{0.05\textwidth}|p{0.35\textwidth}|X|X}
    \textbf{ID} & \textbf{Nome metrica} & \textbf{Valore accettabile} & \textbf{Valore ottimale}  \\
    \hline
    \multicolumn{4}{l}{\cellcolor{primarycolor}\textbf{\textit{Fornitura}}} \\
    \hline
     &  &  &  \\
    \hline
     &  &  &  \\
    \hline
     &  &  &  \\
    \hline
     &  &  &  \\
    \hline
     &  &  &  \\
    \hline
     &  &  &  \\
    \hline
     &  &  &  \\
    \hline
    \multicolumn{4}{l}{\cellcolor{primarycolor}\textbf{\textit{Sviluppo}}} \\
    \hline
     &  &  &  \\
    \hline
     &  &  &  \\
    \end{tabularx}}
   
\subsubsection{Processi di supporto}
    {\renewcommand{\arraystretch}{1.5}
    \begin{tabularx}{\textwidth}{p{0.05\textwidth}|p{0.35\textwidth}|X|X}
    \textbf{ID} & \textbf{Nome metrica} & \textbf{Valore accettabile} & \textbf{Valore ottimale}  \\
    \hline
    \multicolumn{4}{l}{\cellcolor{primarycolor}\textbf{\textit{Verifica}}} \\
    \hline
     &  &  &  \\
    \hline
     &  &  &  \\
    \hline
     &  &  &  \\
    \hline
     &  &  &  \\
    \hline
    \multicolumn{4}{l}{\cellcolor{primarycolor}\textbf{\textit{Gestione della qualità}}} \\
    \hline
     &  &  &  \\
    \end{tabularx}}    
    
\subsubsection{Processi organizzativi}
    {\renewcommand{\arraystretch}{1.5}
    \begin{tabularx}{\textwidth}{p{0.05\textwidth}|p{0.35\textwidth}|X|X}
    \textbf{ID} & \textbf{Nome metrica} & \textbf{Valore accettabile} & \textbf{Valore ottimale}  \\
    \hline
    \multicolumn{4}{l}{\cellcolor{primarycolor}\textbf{\textit{Gestione organizzativa}}} \\
    \hline
     &  &  &  \\
    \end{tabularx}}  

% Qualità di prodotto
\newpage
\section{Qualità di prodotto}
Dopo un'attenta analisi per individuare le proprietà utili per la gestione del ciclo di vita del software 
si è cercato di trovare quali caratteristiche siano necessarie per la realizzazione di un prodotto di 
qualità.
L’acronimo principale di riferimento è MPD, cioè Mean Percentage Difference (MPD), metrica 
utilizzata per valutare la differenza percentuale media tra due valori. Ad esempio, può essere 
utilizzata per calcolare la differenza percentuale media tra i valori di previsione e i valori effettivi di 
una variabile in un modello di previsione o di apprendimento automatico.
In altre parole, l'MPD indica la percentuale media di errore tra le previsioni del modello e i valori 
effettivi della variabile in questione. Un valore di MPD inferiore indica che le previsioni del modello 
sono più precise, mentre un valore più elevato indica una maggiore imprecisione.
\subsection{Documentazione}
    {\renewcommand{\arraystretch}{1.5}
    \begin{tabularx}{\textwidth}{p{0.18\textwidth}|p{0.6\textwidth}|X}
    \textbf{Obiettivo} & \textbf{Descrizione} & \textbf{Metriche}  \\
    \hline
    Correttezza linguistica &  & \\
    \hline
    Leggibilità &  &  \\
    \end{tabularx}}
    
\subsection{Software}
    {\renewcommand{\arraystretch}{1.5}
    \begin{tabularx}{\textwidth}{p{0.18\textwidth}|p{0.6\textwidth}|X}
    \textbf{Obiettivo} & \textbf{Descrizione} & \textbf{Metriche}  \\
    \hline
    Funzionalità &  & \\
    \hline
    Usabilità &  & \\
    \hline
    Efficienza &  &  \\
    \hline
    Affidabilità &  &  \\
    \hline
    Portabilità &  &  \\
    \hline
    Manutenibilità &  &  \\
    \hline
    Copertura dei test &  &  \\
    \end{tabularx}}

\subsection{Metriche utilizzate}
\subsubsection{Documentazione}
    {\renewcommand{\arraystretch}{1.5}
    \begin{tabularx}{\textwidth}{p{0.05\textwidth}|p{0.35\textwidth}|X|X}
    \textbf{ID} & \textbf{Nome metrica} & \textbf{Valore accettabile} & \textbf{Valore ottimale}  \\
    \hline
    \multicolumn{4}{l}{\cellcolor{primarycolor}\textbf{\textit{Correttezza linguistica}}} \\
    \hline
     &  &  &  \\
    \hline
    \multicolumn{4}{l}{\cellcolor{primarycolor}\textbf{\textit{Leggibilità}}} \\
    \hline
     &  &  &  \\
    \end{tabularx}}
    
\subsubsection{Software}
    {\renewcommand{\arraystretch}{1.5}
    \begin{tabularx}{\textwidth}{p{0.05\textwidth}|p{0.35\textwidth}|X|X}
    \textbf{ID} & \textbf{Nome metrica} & \textbf{Valore accettabile} & \textbf{Valore ottimale}  \\
    \hline
    \multicolumn{4}{l}{\cellcolor{primarycolor}\textbf{\textit{Funzionalità}}} \\
    \hline
     &  &  &  \\
    \hline
     &  &  &  \\
    \hline
     &  &  &  \\
    \hline
    \multicolumn{4}{l}{\cellcolor{primarycolor}\textbf{\textit{Usabilità}}} \\
    \hline
     &  &  &  \\
    \hline
     &  &  &  \\
    \multicolumn{4}{l}{\cellcolor{primarycolor}\textbf{\textit{Efficienza}}} \\
    \hline
     &  &  &  \\
    \hline
    \multicolumn{4}{l}{\cellcolor{primarycolor}\textbf{\textit{Affidabilità}}} \\
    \hline
     &  &  &  \\
    \hline
    \multicolumn{4}{l}{\cellcolor{primarycolor}\textbf{\textit{Portabilità}}} \\
    \hline
     &  &  &  \\
    \hline
    \multicolumn{4}{l}{\cellcolor{primarycolor}\textbf{\textit{Manutenibilità}}} \\
    \hline
     &  &  &  \\
    \hline
    \multicolumn{4}{l}{\cellcolor{primarycolor}\textbf{\textit{Copertura dei test}}} \\
    \hline
     &  &  &  \\
    \end{tabularx}}

%Test e specifiche
\newpage
\section{Test e specifiche}
Nella seguente sezione verranno espresse in maniera dettagliata le varie metodologie di test, gli 
obiettivi del testing e i criteri di successo utilizzati durante lo sviluppo del prodotto.
Il gruppo SWEetCode per perseguire la correttezza del prodotto e facilitare la fase di validazione, svolgerà la verifica in parallelo allo sviluppo (Modello a VG).
I test dovranno essere resi il più automatici possibile, per evitare che la fase di testing rallenti la produzione.\\
Lo stato di ciascun test può avere due valori:
\begin{itemize}
\item I: Implementato;
\item NI: Non Implementato.
\end{itemize}

\subsection{Formato ID dei test}
\subsection{Test di unità}
    {\renewcommand{\arraystretch}{1.5}
    \begin{tabularx}{\textwidth}{p{0.12\textwidth}|p{0.7\textwidth}|X}
    \textbf{ID} & \textbf{Descrizione} & \textbf{Stato}  \\
    \hline
     &  & \\
    \hline
     &  &  \\
    \hline
     &  & \\
    \hline
     &  &  \\
    \end{tabularx}}
    
\subsection{Test di integrazione}
    {\renewcommand{\arraystretch}{1.5}
    \begin{tabularx}{\textwidth}{p{0.12\textwidth}|p{0.7\textwidth}|X}
    \textbf{ID} & \textbf{Descrizione} & \textbf{Stato}  \\
    \hline
     &  & \\
    \hline
     &  &  \\
    \hline
     &  & \\
    \end{tabularx}}

\subsection{Test di sistema}
    {\renewcommand{\arraystretch}{1.5}
    \begin{tabularx}{\textwidth}{p{0.12\textwidth}|p{0.55\textwidth}|p{0.12\textwidth}|X}
    \textbf{ID} & \textbf{Descrizione} & \textbf{Requisito} & \textbf{Stato}  \\
    \hline
     &  & \\
    \hline
     &  &  \\
    \end{tabularx}}

\subsection{Test di accettazione}
    {\renewcommand{\arraystretch}{1.5}
    \begin{tabularx}{\textwidth}{p{0.12\textwidth}|p{0.7\textwidth}|X}
    \textbf{ID} & \textbf{Descrizione} & \textbf{Stato}  \\
    \hline
     &  & \\
    \hline
     &  &  \\
    \end{tabularx}}
    
\section{Resoconto delle attività di verifica}

\subsection{Verifica della documentazione}
\subsubsection{Errori ortografici}
\subsubsection{Indici di Gulpease}

\subsection{Verifica dei processi}
\subsubsection{Estimated at completion}
\subsubsection{Budget variance e schedule variance}
\subsubsection{Actual cost e estimate to complete}
\subsubsection{Earned value e planned value}
\subsubsection{Requirements stability index e satisfied obligatory requirements}
\subsubsection{Code coverage back-end}
\subsubsection{Code coverage front-end}
\subsubsection{Passed test cases percentage}
\subsubsection{Failed test cases percentage}
\subsubsection{Comprensibilità del codice}

\end{document}
