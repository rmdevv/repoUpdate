\documentclass[10pt, a4paper]{article}

% Parametri che modificano il file main.tex
% Le uniche parti da cambiare su main.tex sono:
% - vari \vspace tra sezioni
% - tabella azioni da intraprendere
% - sezione altro

\def\data{2023-11-15}
\def\oraInizio{15:00}
\def\oraFine{16:00}
\def\luogo{Piattaforma Google Meet}

\def\tipoVerb{Esterno} % Interno - Esterno

\def\nomeResp{Feltrin E.} % Cognome N.
\def\nomeVer{Campese M.} % Cognome N.
\def\nomeSegr{Orlandi G..} % Cognome N.

\def\nomeAzienda{Azzurro Digitale}
\def\firmaAzienda{azzurrodigitale.png}
\def\firmaResp{emanuele.png} % nome Responsabile

\def\listaPartInt{
Bresolin G.,
Campese M.,
Ciriolo I.,
Dugo A.,
Feltrin E.,
Michelon R.,
Orlandi G.
}

\def\listaPartEst{
\textit{Azzurro Digitale:},
Caliendo G.,
D'Avanzo C.}

}

\def\listaRevisioneAzioni {x}

\def\listaOrdineGiorno {
{Riunione con l'azienda AzzurroDigitale;},
{L'azienda ci ha esposto i requisiti che deve avere il progetto e ci hanno fatto una breve introduzione per discutere in modo generale , cosa si aspettano il software debba saper fare;},
{Possibilità di porre alcune domande generiche sul progetto ai referenti Caliendo G. e D'Avanzo C., con seguente approfondimento dei vari temi toccati.\\ .}
}

\def\listaDiscussioneInterna {
{Discussione 1;},
{Discussione 2;},
}

\newcommand{\domris}[2]{\textbf{#1}\\#2}

\def\listaDiscussioneEsterna {
\domris
{In generale , quali sono i requisiti funzionali fondamentali che deve avere il sistema?
}
{L'azienda ci ha risposto , che certamente la ricezione in input e la gestione dei documenti , deve essere efficiente;
},
\domris
{Quali sono le funzioni principali per il caricamento dei documenti?
}
{L'azienda ha detto che come minimo , il sistema dovrà essere capace di ricevere in input ed elaborare documenti Pdf e documenti Word;
},
\domris
{I documenti caricati dovranno rimanere accessibili?
}
{L'azienda ci ha risposto che i documenti caricati potranno essere accessibili , ma che non serviranno grandi cose , basta un catalogo ;
},
\domris
{Quali funzioni serviranno per eliminare i documenti? Il modello dovrà essere riaddestrato?
}
{L'azienda ci ha risposto che se un documento subisce modifiche , il sistema dovrà evolvere ed essere riaddestrato;
},
\domris
{L'interazione tra l'utente e il chatbot , quali saranno i protocolli? 
}
{L'azienda ci ha risposto , che per quanto riguarda il comando vocale , da parte dei lavoratori non sarà fondamentale , ma che come minimo se un lavoratore chiede cose errate , il sistema dovrà dirglielo e che possa memorizzare una storia delle chat tra il lavoratore ed esso}.
}
} 




\def\listaDecisioni {x}

\usepackage{style}
\usepackage{headerfooter}

\title{\titolo}
\author{SWEetCode}

\begin{document}

% PRIMA PAGINA
\include{firstpage}

% REGISTRO DELLE VERSIONI
%Registro in ordine dalla più recente alla meno recente!
{\renewcommand{\arraystretch}{1.5}
\section*{Registro delle versioni}

\begin{xltabular}{\textwidth}{c|c|c|c|X}
\label{tab:long}

\textbf{Versione} & \textbf{Data} & \quantities{\textbf{Responsabile di}\\\textbf{stesura}}& \textbf{Revisore} & \quantities{\textbf{Dettaglio e}\\\textbf{motivazioni}} \\
\endfirsthead

\textbf{Versione} & \textbf{Data} & \quantities{\textbf{Responsabile di}\\\textbf{stesura}}& \textbf{Revisore} & \quantities{\textbf{Dettaglio e}\\\textbf{motivazioni}} \\
\endhead

\multicolumn{5}{r}{{Continua nella pagina successiva}} \\
\endfoot

\endlastfoot

\hline
v2.0.0(1) & $2023-11-18$ & Michelon R. & Campese M. & Aggiornata introduzione.\\
\hline
v1.0.0(8) & $2023-11-18$ & Michelon R. & Campese M. & Cambiata introduzione.\\
\hline
v1.0.0(6) & $2023-11-17$ & Dugo A. & Riccardo & modifica per prova1.\\
\hline
v1.0.0(5) & $2023-11-17$ & Dugo A. & Campese M. & registro delle modifiche prova.\\
\hline
v1.0.1 & $2023-11-14$ & \quantities{Ciriolo I.} & Campese M. &  Studio dei primi casi d'uso.\\
\hline
v0.0.1 & $2023-11-06$ & \quantities{Ciriolo I.\\Orlandi G.} & Campese M. &  Impostazione del documento, Introduzione.\\
    
\end{xltabular}}
\newpage

% INDICE
\tableofcontents
% SOMMARIO
\section*{Sommario}
\subsection*{Elenco delle immagini}
% Es.Immagine 1: Didascalia.
\subsection*{Elenco dei grafici}
% Es.Grafico 1: Didascalia.
\subsection*{Elenco delle tabelle}
% Es.Tabella 1: Didascalia.
\newpage

% INIZIO PAGINE

\section{Introduzione}



\subsection{Scopo del documento}
Il documento ha l'obiettivo di definire le risorse da impiegare, le modalità e le tempistiche da seguire per lo svolgimento del progetto. In particolare viene effettuata un'analisi dei rischi attesi, a cui vengono affiancate delle pratiche di mitigazione degli stessi. Si propone inoltre una valutazione dell'efficacia di queste pratiche,così da portare ad eventuali miglioramenti o correzioni delle stesse nel caso in cui non dovessero portare ai risultati desiderati.

Il documento si sviluppa poi nelle sezioni di pianificazione delle attività, indicando preventivo e consuntivo di ogni periodo e infine si conclude con la parte di retrospettiva, in cui vengono analizzate la gestione del tempo e del budget e le pratiche che si sono rivelate più o meno buone nel corso dello svolgimento del progetto.

In aggiunta è necessario specificare che tale documento viene redatto con un approccio incrementale, in maniera tale da poter implementare facilmente dei cambiamenti nel corso del tempo a seconda delle necessità.

\subsection{Scopo del prodotto}

\subsection{Glossario (da inizializzare )}

\subsection{Riferimenti}

 %SEZIONE RIF. NORMATIVI
\subsubsection{Riferimenti normativi} 
\begin{itemize}
\item \textit{Regolamento del progetto didattico}: \\
\href{https://www.math.unipd.it/~tullio/IS-1/2023/Dispense/PD2.pdf}{https://www.math.unipd.it/~tullio/IS-1/2023/Dispense/PD2.pdf}\\
(Ultimo accesso: $2023-11-13$);
\item \textit{Norme di progetto v0.4.3};
\item \textit{Analisi dei requisiti v1.0.1}.
\end{itemize}

% SEZIONE RIF. INFORMATIVI
\subsubsection{Riferimenti informativi}
\begin{itemize}
    \item \textit{Glossario v0.0.1} (da creare parallelamente) specificare la versione oppure come risorsa "web" per facilitare la consultazione rapida e agile quindi con data di ultimo accesso?
   
    \item \textit{Presentazione capitolato}:\\
    \href{https://www.math.unipd.it/~tullio/IS-1/2023/Progetto/C1.pdf}{https://www.math.unipd.it/~tullio/IS-1/2023/Progetto/C1.pdf}\\
    (Ultimo accesso: $2023-11-13$);   
    
    \item \textit{Verbali esterni ed interni};
    
    \item \textit{Dispense su ciclo di vita del SW}:\\
    \href{https://www.math.unipd.it/~tullio/IS-1/2023/Dispense/T2.pdf}{https://www.math.unipd.it/~tullio/IS-1/2023/Dispense/T2.pdf}\\
    (Ultimo accesso: $2023-11-13$);
    
    \item  \textit{Dispense su gestione di progetto}:\\
    \href{https://www.math.unipd.it/~tullio/IS-1/2023/Dispense/T4.pdf}{https://www.math.unipd.it/~tullio/IS-1/2023/Dispense/T4.pdf}\\
    (Ultimo accesso: $2023-11-13$).
\end{itemize}


\section{Calendario di massima del progetto}

\subsection{Introduzione}
\color{gray}(preventivo  della lettera di candidatura e stima completamento lavori sempre della candidatura)
\color{black}

\section{Stima dei costi di realizzazione}

\section{Rischi e loro mitigazione}
\paragraph{}Questa sezione si occupa di analizzare le difficoltà che si possono incontrare durante lo svolgimento del progetto e che possono complicare la pianificazione delle attività, portando a rallentamenti e ostacoli nell'avanzamento.\\
Per poter individuare e gestire questi rischi, vengono di seguito esaminati e corredati da descrizione, previsione della loro occorrenza e grado di pericolosità e infine da misure di mitigazione degli effetti negativi nel caso si verifichino.

\color{gray}\paragraph{ idee} (L'analisi dei rischi deve essere seguita da "gestione dei rischi" e da "valutazione efficacia delle pratiche di mitigazione" così da portare ad eventuali miglioramenti o correzioni delle stesse se non portano i risultati desiderati.)\\
Possibile quindi la necessità di una tabella di feedback con seguenti indici:
\begin{enumerate}
    \item l’occorrenza di rischi
    \item l’attuazione delle misure di mitigazione previste
    \item la valutazione del loro esito
\end{enumerate}
   
\paragraph{Valutazione efficacia delle misure di mitigazione} questa sezione può essere inserita nella sezione 8 di retrospettiva, comunque da segnalare qui dove si possono trovare i feedback.
\paragraph{Cruscotto} il cruscotto di valutazione permette di individuare
facilmente lo stato di completamento delle attività. Esso va aggiornato quanto più spesso possibile perché mandi segnali/notifiche utili per guidare il lavoro futuro, quindi non solo ad ogni rilascio esterno dei prodotti ma ad ogni caricamento in repository di artefatti verificati(per ogni documento?).\\
\color{black}

% TABELLA RISCHI

Prova tabella orizzontale





{\renewcommand{\arraystretch}{1.5}
\begin{tabularx}{\textwidth}{>{\columncolor{primarycolor}}c|X}
%\caption{Esempio tabella rischio}
\textbf{ID Rischio} & Esempio \\
\hline
\textbf{Rischio} &  \\
\hline
\textbf{Descrizione} & \\
\hline
\textbf{Occorrenza} & \\
\hline
\textbf{Impatto} & \\
\hline
\textbf{Misure di mitigazione} & \\
\end{tabularx}

%PERSONALI

\subsection{Rischi personali}

% RISCHIO IMPEGNI PERSONALI

\subsubsection{Impegni e problemi personali}
{\renewcommand{\arraystretch}{1.5}
\begin{tabularx}{\textwidth}{c|X}
%\caption{Esempio tabella rischio}
\textbf{ID Rischio} & Esempio \\
\hline
\textbf{Rischio} & Impegni e problemi personali\\
\hline
\textbf{Descrizione} & Ogni componente del team ha impegni esterni e può avere problemi strettamente personali. Questo indica che qualche membro può non essere disponibile in certi momenti.\\
\hline
\textbf{Occorrenza} & Media\\
\hline
\textbf{Impatto} & ALto\\
\hline
\textbf{Misure di mitigazione} & I membri interessati si impegnano ad avvisare tempestivamente il gruppo che per far fronte a tale rischio coprrà l’intervallo non produttivo del componente con una suddivisione omogenea tra i restanti colleghi delle attività rimaste in sospeso.
Riuscire a non spostare la \textit{milestone} è prioritario.\\
\end{tabularx}}




\subsubsection{Problemi fra componenti del gruppo}

{\renewcommand{\arraystretch}{1.5}
\begin{tabularx}{\textwidth}{c|X}
%\caption{Esempio tabella rischio}
\textbf{ID Rischio} & Esempio \\
\hline
\textbf{Rischio} & Problemi fra componenti del gruppo  \\
\hline
\textbf{Descrizione} & \\
\hline
\textbf{Occorrenza} & \\
\hline
\textbf{Impatto} & \\
\hline
\textbf{Misure di mitigazione} & \\
\end{tabularx}

\begin{itemize}
    \item Carico di lavoro personale eccessivo;\\ 
     
\end{itemize}

%ORGANIZZATIVI

\subsection{Rischi organizzativi}

\subsubsection{Carico di lavoro distribuito in maniera non equilibrata}

{\renewcommand{\arraystretch}{1.5}
\begin{tabularx}{\textwidth}{c|X}
%\caption{Esempio tabella rischio}
\textbf{ID Rischio} & Esempio \\
\hline
\textbf{Rischio} & Carico di lavoro distribuito in maniera non equilibrata \\
\hline
\textbf{Descrizione} & \\
\hline
\textbf{Occorrenza} & \\
\hline
\textbf{Impatto} & \\
\hline
\textbf{Misure di mitigazione} & \\
\end{tabularx}

\subsubsection{Sottostima del tempo necessario per una attività}

{\renewcommand{\arraystretch}{1.5}
\begin{tabularx}{\textwidth}{c|X}
%\caption{Esempio tabella rischio}
\textbf{ID Rischio} & Esempio \\
\hline
\textbf{Rischio} & Sottostima del tempo necessario per una attività\\
\hline
\textbf{Descrizione} & Il team può andare in contro ad una sottostima del tempo necessario per il completamento di un requisito o di un' attività.\\
\hline
\textbf{Occorrenza} & Alta\\
\hline
\textbf{Impatto} & Alto\\
\hline
\textbf{Misure di mitigazione} & tale errore di valutazione deve essere reso noto al team nel modo più rapido possibile; chi ha disponibilità di tempo viene incaricato a fornire assistenza ai colleghi per minimizzare il ritardo nel completamento dell'obiettivo.\\

\end{tabularx}}

{\renewcommand{\arraystretch}{1.5}
\begin{tabularx}{\textwidth}{c|X}
%\caption{Esempio tabella rischio}
\textbf{ID Rischio} & Esempio \\
\hline
\textbf{Rischio} & Stima errata dei costi \\
\hline
\textbf{Descrizione} & \\
\hline
\textbf{Occorrenza} & \\
\hline
\textbf{Impatto} & \\
\hline
\textbf{Misure di mitigazione} & \\
\end{tabularx}

{\renewcommand{\arraystretch}{1.5}
\begin{tabularx}{\textwidth}{c|X}
%\caption{Esempio tabella rischio}
\textbf{ID Rischio} & Esempio \\
\hline
\textbf{Rischio} & Disponibilità di lavoro non sfruttato \\
\hline
\textbf{Descrizione} & \\
\hline
\textbf{Occorrenza} & \\
\hline
\textbf{Impatto} & \\
\hline
\textbf{Misure di mitigazione} & \\
\end{tabularx}


\begin{itemize}
    \item Carico di lavoro distribuito in modo non equilibrato;
    \item Modifica requisiti in corso;
    \item Ritardo di consegna (stima errata delle tempistiche);
    \item Stima errata dei costi;
    \item Disponibilità di lavoro non sfruttato.
\end{itemize}

%TECNOLOGICI

\subsection{Rischi tecnologici}
\begin{itemize}
    \item Scarsa esperienza tecnologica;
    \item Guasti hardware e problematiche software
\end{itemize}
\begin{itemize}
    \item Inesperienza con le tecnologie/tool di lavoro
    \item
\end{itemize}
%\il pdp è sia una linea guida per il progetto che un suo resoconto, si continua ad aggiornarlo fino alla fine del progetto.


\section{Pianificazione}
\color{gray}\paragraph{Sprint o Periodi}
(descrizione suddivisione periodi in base a cosa? impostare la suddivisione sugli sprint o periodi (11 giorni come la turnazione dei ruoli?).
\paragraph{}ho usufruito delle ore di scriba mentre avevo delle ore "libere" nonostante abbia ruolo di verificatore;
\color{black}
\subsection{Requirements and Technology Baseline}

\subsection{Product Baseline}


\section{Preventivo}




\section{Consuntivo}

\section{Retrospettiva generale}
\subsection{Gestione delle risorse}
\subsubsection{Tempo}
\subsubsection{Budget}
\subsection{Aspetti positivi}
\subsection{Aspetti negativi}
\color{gray}\paragraph{}pratica del vincolo di un ruolo, può non far sfruttare tutta la disponibilità di lavoro dei componenti, etc.
\color{black}
\subsection{Conclusioni}

\end{document}