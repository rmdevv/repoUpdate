\documentclass[10pt, a4paper]{article}

% Parametri che modificano il file main.tex
% Le uniche parti da cambiare su main.tex sono:
% - vari \vspace tra sezioni
% - tabella azioni da intraprendere
% - sezione altro

\def\data{2023-11-15}
\def\oraInizio{15:00}
\def\oraFine{16:00}
\def\luogo{Piattaforma Google Meet}

\def\tipoVerb{Esterno} % Interno - Esterno

\def\nomeResp{Feltrin E.} % Cognome N.
\def\nomeVer{Campese M.} % Cognome N.
\def\nomeSegr{Orlandi G..} % Cognome N.

\def\nomeAzienda{Azzurro Digitale}
\def\firmaAzienda{azzurrodigitale.png}
\def\firmaResp{emanuele.png} % nome Responsabile

\def\listaPartInt{
Bresolin G.,
Campese M.,
Ciriolo I.,
Dugo A.,
Feltrin E.,
Michelon R.,
Orlandi G.
}

\def\listaPartEst{
\textit{Azzurro Digitale:},
Caliendo G.,
D'Avanzo C.}

}

\def\listaRevisioneAzioni {x}

\def\listaOrdineGiorno {
{Riunione con l'azienda AzzurroDigitale;},
{L'azienda ci ha esposto i requisiti che deve avere il progetto e ci hanno fatto una breve introduzione per discutere in modo generale , cosa si aspettano il software debba saper fare;},
{Possibilità di porre alcune domande generiche sul progetto ai referenti Caliendo G. e D'Avanzo C., con seguente approfondimento dei vari temi toccati.\\ .}
}

\def\listaDiscussioneInterna {
{Discussione 1;},
{Discussione 2;},
}

\newcommand{\domris}[2]{\textbf{#1}\\#2}

\def\listaDiscussioneEsterna {
\domris
{In generale , quali sono i requisiti funzionali fondamentali che deve avere il sistema?
}
{L'azienda ci ha risposto , che certamente la ricezione in input e la gestione dei documenti , deve essere efficiente;
},
\domris
{Quali sono le funzioni principali per il caricamento dei documenti?
}
{L'azienda ha detto che come minimo , il sistema dovrà essere capace di ricevere in input ed elaborare documenti Pdf e documenti Word;
},
\domris
{I documenti caricati dovranno rimanere accessibili?
}
{L'azienda ci ha risposto che i documenti caricati potranno essere accessibili , ma che non serviranno grandi cose , basta un catalogo ;
},
\domris
{Quali funzioni serviranno per eliminare i documenti? Il modello dovrà essere riaddestrato?
}
{L'azienda ci ha risposto che se un documento subisce modifiche , il sistema dovrà evolvere ed essere riaddestrato;
},
\domris
{L'interazione tra l'utente e il chatbot , quali saranno i protocolli? 
}
{L'azienda ci ha risposto , che per quanto riguarda il comando vocale , da parte dei lavoratori non sarà fondamentale , ma che come minimo se un lavoratore chiede cose errate , il sistema dovrà dirglielo e che possa memorizzare una storia delle chat tra il lavoratore ed esso}.
}
} 




\def\listaDecisioni {x}

\usepackage{style}
\usepackage{headerfooter}

\title{\data}
\author{SWEetCode}

\begin{document}

% PRIMA PAGINA
\include{firstpage}

% INIZIO PAGINE
\columnratio{0.31}
\setlength{\columnsep}{2.2em}
\setlength{\columnseprule}{4pt}
\colseprulecolor{lightcol}
\begin{paracol}{2}

% 11111111111111111111111111111111111111111111111111111111111111111111111111111

\intestazione
\vspace{5.0em}

\partecipanti

\newpage

\switchcolumn
\revisioneAzioni
\vspace{23.5em}

\ordineGiorno

\newpage

% 22222222222222222222222222222222222222222222222222222222222222222222222222222

\switchcolumn

\heading{Discussione}\\
\textbf{Sintesi degli argomenti\\discussi}

~\newpage
%~ \\\\\\\\\\\\\\\~

\heading{Decisioni}\\
\textbf{Decisioni prese durante\\la discussione}

\switchcolumn

\discussione
Successivamente alla discussione, il gruppo ha deciso di iniziare subito con l'analisi dei requisiti, abbiamo scelto il sito che facesse al caso nostro, per realizzare i diagrammi dei Casi d'uso .

L'azienda ha ribadito che il primo blocco di lavoro sarà di 6 settimane, entro le quali sarà richiesto un PoC (con scadenza entro il 2023-12-06), nel quale deve essere presente una solida Analisi dei requisiti. 

Viene infine specificato che sarà richiesto di elaborare un'analisi dei costi di costruzione e di mantenimento del progetto.
Inoltre abbiamo deciso di cambiare gli ambienti di lavoro, con \textit{framework} che fossero più adatti ai nostri scopi, come ad esempio passare da overleaf , per la stesura dei documenti , a Jira software o passare da Github per la repository dei documenti , a Vsual StudioCode .
Infine, abbiamo deciso di studiare seriamente i casi d'uso , che l'azienda ci ha proposto .

\vspace{12.5em}

\decisioni

\end{paracol}

%\newpage
\vspace{7em}
\heading{Azioni da Intraprendere}

{\renewcommand{\arraystretch}{1.5}
\begin{tabularx}{\textwidth}{c|X|c|c|c}
\textbf{ID Issue} & \textbf{Azione} & \textbf{Incaricato} & \textbf{Revisore} & \textbf{Scadenza} \\
\hline
\#36 &
Stesura verbale riunione &
Orlandi G. &
Campese M. &
$2023-11-15$ \\
\hline
- &
Lavorare su Analisi dei requisiti & 
Ciriolo I. Orlandi G. &
Campese M. &
$2023-11-15$ \\
\hline
- &
Automazione versioni documenti e aggiornamento norme di progetto &
Team &
Team &
$2023-11-15$ \\
\hline
- &
Iniziare Piano di Qualifica &
Feltrin E. Orlandi G. &
Campese M. &
$2023-11-15$ \\
\hline
- &
 &
Feltrin E. &
Campese M. &
$2023-11-15$ \\
\end{tabularx}}

\vspace{3em}

\heading{Altro}

\textbf{Prossima Riunione}

La prossima riunione con AzzurroDigitale è pianificata il 2023-11-22 alle 15:00 , per un approfondimento riguardo l'Analisi dei requisiti.

\end{document}